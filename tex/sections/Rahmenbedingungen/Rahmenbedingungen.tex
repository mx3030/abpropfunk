\documentclass[../main.tex]{subfiles}
\graphicspath{{\subfix{../img/}},{img},{img/ink}}
\begin{document}
\section{Rahmenbedingungen}
Bei der Planung der Doppelstunde wird von einer sehr heterogenen Klasse ausgegangen. Das bedeutet, dass die SuS sehr unterschiedliche Interessensgebiete haben und ihre mathematischen Fähigkeiten unterschiedlich stark ausgeprägt sind.\\
Die ersten 45 Minuten der Doppelstunde führen die SuS in Gruppenarbeit (2-3 SuS pro Gruppe) verschiedene Experimente durch. Die Materialien werden an den verschiedenen Stationen bereitgestellt. Zur Durchführung verwenden die SuS ein iPad. Über eine Website (z.B. Moodle) greifen die SuS auf interaktive Arbeitsblätter zu, mit denen sie durch den Versuch und die Bearbeitung geführt werden können.\\
Im zweiten Teil der Doppelstunde präsentieren einzelne Gruppen zunächst ihre Ergebnisse mit Beamer und HDMI-Kabel. Durch die vorbereiteten Arbeitsblätter ist für die Präsentation wenig Arbeit erforderlich. Die Ergebnisse werden diskutiert und die wichtigsten Beobachtungen an der Tafel notiert.\\
\begin{tcolorbox}[
    title= {\centering Materialien},
    title filled=false, 
    colback=white, 
    colframe=white!80!black, 
    coltitle=black, 
    arc=0pt,
    outer arc=0pt
]
\colorbox{white}{\hspace{3cm}}
\begin{minipage}{0.8\textwidth}  % Adjust the width as needed (0.8\textwidth in this example)
\begin{itemize}
    \item Materialien für Experimente
    \item iPad
    \item Beamer + HDMI-Kabel
    \item Tafel
\end{itemize}
\end{minipage}

\end{tcolorbox}

\end{document}
