\documentclass[../main.tex]{subfiles}
\graphicspath{{\subfix{../img/}},{img},{img/ink}}
\begin{document}
\section{Methodische Überlegungen}
Die Heterogenität der Klasse stand im Mittelpunkt der methodischen Überlegungen. Die Umsetzung der Experimentierphase erfolgt im Grunde nach dem bekannten Stationenlernen. Die Problemstellungen der Stationen behandeln dabei ganz verschiedene Bereiche des Alltagslebens. Dadurch soll versucht werden, dass jede Gruppe eine Station findet, die sich auch mit dem eigenen Interesse deckt. Zusätzlich wird darauf geachtet, dass die Stationen räumlich möglichst getrennt voneinander stattfinden. Dadurch soll Rücksicht auf SuS genommen werden, die eher ruhigere Arbeitsumgebungen bevorzugen.\\
Um den stärken Gruppen ein möglichst unabhängiges Arbeiten zu ermöglichen, werden interaktive Arbeitsblätter eingesetzt. Diese leiten schrittweise durch das Experiment. Dabei erhalten die SuS immer wieder Rückmeldungen, ob entsprechende Teillösungen korrekt sind. Dadurch kann die Konzentration der Lehrkraft auf schwächeren Gruppen liegen, während die anderen selbstständig arbeiten können. Für diese schwächeren Gruppen stehen zusätzlich Hilfebuttons bereit, die umfangreichere Erklärungen oder Tipps liefern.\\
Da die gesamte Bearbeitung digital stattfindet, ist sowohl Struktur und Ordentlichkeit der Arbeitsschritte gewährleistet. Die SuS können also die Konzentration ganz auf das Verständnis des Modellierungsprozesses legen. Um trotzdem eine gewisse Individualität in der Lösung herzustellen, ist es denkbar, dass starke Gruppen mit Hilfe von Bildschirmfotos ein kompakteres Plakat der Versuchsergebnisse erstellen.\\
Nach dem Experimentieren, sollten alle Gruppen mindestens eine Problemstellung grafisch gelöst haben. Es erfolgt dann der Übergang in die Präsentationsphase. Dabei sollen verschiedene Gruppen kurz ihre Ergebnisse vorstellen. Haben die Gruppen kein Plakat erstellt, kann ohne Problem auch direkt das interaktive Arbeitsblatt verwendet werden. Um einen Denkprozess bei den SuS anzuregen, wird darauf geachtet, dass neben proportionalen Vorgängen auch nicht-proportionale Vorgänge vorgestellt werden.\\
Nach jeder Präsentation wird im Gespräch mit der ganzen Klasse diskutiert, welche zentralen Beobachtungen und Erkenntnisse an der Tafel festgehalten werden soll. Um die SuS in die passende Richgung zu lenken sind folgende Impulse denkbar:
\begin{itemize}
    \item Welche Besonderheit wurde bei diesem Vorgang ausgenutzt? (Messpunkte liegen auf Gerade durch den Ursprung.)
    \item Welche Annahme wurde bei der Bestimmung der Lösung verwendet? (Auch die weiteren Punkte würden im Bereich der Gerade liegen.)
    \item Warum konnte hier keine Vorhersage getroffen werden? (Weil die Messpunkte auf keiner Gerade liegen.)
    \item Worin liegt der Unterschied zwischen den Geraden? (Eine Gerade geht durch den Ursprung.)
    \item Was ist der Vorteil, wenn die Gerade durch den Ursprung läuft? (Man kann auch den Dreisatz andwenden.)
\end{itemize}
Es werden nicht alle Gruppen ihre Ergebnisse und speziellen Experimente präsentieren können. Damit trotzdem alle Gruppen die entsprechende Wertschätzung erfahren, wird klar gemacht, dass die Ergebnisse online zur Verfügung gestellt werden (z.B. als PDF-Dateien auf CryptPad) und im Rahmen der weiteren Unterrichtseinheit immer wieder darauf zurückgegriffen wird. Das erfolgt sowohl im Rahmen späterer Unterrichtsstunden als auch in Form von Hausaufgaben.\\
Nachdem ca. 20 Minuten präsentiert wurde, werden die Beobachtungen der SuS zu einem systematischen Tafelaufschrieb zusammengefasst und gegebenenfalls ergänzt. Dieser wird so gestaltet, dass in der nächsten Stunde die Idee der Quotientengleichheit in das Schaubild eingefügt werden und der Begriff des Proportionalitätsfaktors, sowie das Vorgehen beim Dreisatz ergänzt werden kann.
\end{document}
