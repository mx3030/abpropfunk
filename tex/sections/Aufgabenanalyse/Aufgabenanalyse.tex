\documentclass[../main.tex]{subfiles}
\graphicspath{{\subfix{../img/}},{img},{img/ink}}
\begin{document}
\section{Aufgabenanalyse} \label{section:Materialien}
Für den ersten Teil der Einführungsstunde, setzten sich die SuS mit einer Reihe von Problemstellungen auseinander. Diese werden durch Experimentieren gelöst und in Form von Stationen aufgebaut. Jede Station beschäftigt sich mit einer konkreten Fragestellung. Um der Heterogenität gerecht zu werden, wird versucht unterschiedlichste Interessensgebiete abzudecken. Die interaktiven Arbeitsblätter können unter 
\begin{center}
    https://mx3030.github.io/abpropfunk/
\end{center} 
eingesehen werden. Im Folgenden sind die 7 Stationen aufgeführt, in denen verschiedene Zuordnungen untersucht werden können. Hauptsächlich handelt es sich dabei um proportionale und lineare Zuordnungen.
\begin{enumerate}
    \item \underline{Der tropfende Wasserhahn} \\
        Wie viel Wasser wird bei einem tropfenden Wasserhahn an einem Tag verschwendet?\\
        Die SuS untersuchen hier die proportionale Zuordnung Zeit $\mapsto$ Volumen.\\
       \colorbox{tablegray1}{Vertiefung:} Wie breit ist ein Wassertropfen?
    \item \underline{Strategisches Zählen} \\
        Wie viele Seiten Papier sind in einem Papierstapel?\\ 
        Hier sind zwei proportionale Zuordnungen denkbar. Entweder Gewicht $\mapsto$ Seitenanzahl oder Höhe $\mapsto$ Seitenanzahl.\\
        \colorbox{tablegray1}{Vertiefung:} Ausweitung der Betrachtung auf andere Materialien (Geld zählen, Schrauben/Muttern in Box zählen, \ldots). 
    \item \underline{Kabelsalat}\\
        Wie lang ist das Seil/Verlängerungskabel?\\ 
        Die Lösung erhält man über die proportionale Zuordnung Gewicht $\mapsto$ Länge.\\ 
        \colorbox{tablegray1}{Vertiefung:} Wie viel wiegt die Kabeltrommel ohne Kabel?
    \item \underline{Löse den Mordfall} \\
        Wie lange dauert es, bis die Kerze abgebrannt ist?\\
        In diesem Fall liegt eine lineare Zuordnung vor, auch wenn nicht vom Ursprung aus gestartet wird. Für die Lösung nutzt man die Zuordnung Gewicht $\mapsto$ Höhe.\\
        \colorbox{tablegray1}{Vertiefung:}  Die Kerze hatte ursprünglich eine bestimmte Höhe. Wie lange brennt die Kerze schon?  
    \item \underline{Getränke kühlen} \\
        Wie verändert sich das Abkühlen von kochendem Wasser bei der Zugabe von Salz?\\
        Mit Hinblick auf die spätere Diskussion wird hier eine nichtlineare Zuordnung untersucht.
    \item \underline{Gewindestange}\\
        Die SuS berechnen die Anzahl an Umdrehungen, um eine Schraube auf eine Gewindestange zu drehen.\\
        Je nach Fragestellung lässt sich diese Station als lineare oder proportionale Zuordnung gestalten. Dabei wird Umdrehungen $\mapsto$ Länge zugeordnet.\\
        \colorbox{tablegray1}{Vertiefung:} Wie viele Umdrehungen kann ein Bleistift in einem Spitzer benutzt werden?
    \item \underline{Hebelkraft}\\
        Wie funktioniert ein Hebel?\\
        Bei diesem Experiment wird die Zuordnung Länge $\mapsto$ Gewicht untersucht.\\
        \colorbox{tablegray1}{Vertiefung:} Wo werden Hebel eingesetzt?
\end{enumerate}
\end{document}
