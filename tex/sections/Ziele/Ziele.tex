\documentclass[../main.tex]{subfiles}
\graphicspath{{\subfix{../img/}},{img},{img/ink}}
\begin{document}
\section{Kompetenzen und Ziele}
Das Schwerpunktziel der Unterrichtsstunde ist es, dass die SuS eine Sinnhaftigkeit in der Auseinandersetzung mit dem Funktionsbegriff erkennen. Die Problemstellungen der Experimente sind so gewählt, dass Alltagsfragen systematisch untersucht werden können. Am Ende soll allen klar sein, dass man proportionale Vorgänge durch eine Ursprungsgerade beschreiben kann und damit eine grafische Lösung für verschiedene Fragestellungen möglich ist.\\ 
Die Kompetenzen die dazu erforderlich sind, sind sowohl theoretischer, als auch praktischer Natur. Letzteres wird durch den guten Aufbau der Versuche und die korrekte Durchführung der Messungen erreicht. \\
Die theoretischen Anforderungen liegen im mathematischen und physikalischen Bereich. Das Messen und Umrechnen von Einheiten ist bei der Bearbeitung von zentraler Wichtigkeit. Diese Fähigkeit wird in der 7. Klasse auch mit dem Beginn des Physikunterrichts vertieft. Die SuS sollen hier auch die starke Verbindung der beiden Fächer erkennen.\\
Schaut man in die prozessbezogenen Kompetenzen des Bildungsplans Mathematik, so liegt der Fokus in dieser Doppelstunde auf dem Modellieren und dem Kommunizieren. Wie bereits im vorherigen Kapitel verdeutlicht, findet an dieser Stelle aber eine didaktische Reduktion statt, sodass sich die Kompentenzen allein auf das Verstehen der systematischen Herangehensweise zum Lösen von solchen Problemen beschränkt. D.h. das Problem lösen als Kompetenz wird weniger gefördert, da die interaktiven Arbeitsblätter einen sinnvollen Lösungsweg vorgeben.\\
Die Heterogenität im Bereich der mathematischen Fähigkeiten zeigt sich am deutlichsten in den Rechnenfertigkeiten. Diese rücken aus diesem Grund erst am Ende der interaktiven Arbeitsblätter stärker in der Vordergrund.\\
Die Verwendung interaktiver Arbeitsblätter liefert zusätzlich die Möglichkeit der Verwendung von GeoGebra-Applets. Diese Software ist in späteren Schuljahren ein hervorragendes Hilfsmittel für den Umgang mit komplizierten Funktionen und der Differential- bzw. Integralrechnung. Da sich die Applets im Voraus in ihrem Umfang speziell anpassen lassen, kann den SuS ein langsamer Einstieg in die Möglichkeiten des Programms ermöglicht werden.\\
In Tabelle \ref{table:2} sind die einzelnen Teilschritte zum Erreichen des Schwerpunktziels aufgeführt.\\
Bei stärkeren SuS ist es auch denkbar, dass zum Lösen der vertiefenden Aufgabenstellungen ChatGPT zum Einsatz kommt. Das interaktive Arbeitsblatt liefert in diesem Fall nur die Problemstellung und die SuS erarbeiten in der Diskussion mit ChatGPT einen möglichen Lösungsweg.\\

\begin{table}[h]
    \centering
    \begin{tabularx}{\textwidth}
        { 
            >{\centering\arraybackslash}p{0.2\linewidth} 
            %p{0.3\linewidth} 
            @{\hspace{0.5cm}} 
            %>{\centering\arraybackslash}p{0.3\linewidth} 
            p{0.3\linewidth} 
            @{\hspace{0.5cm}} 
            %>{\centering\arraybackslash}p{0.3\linewidth} 
            p{0.4\linewidth} 
            @{\hspace{0.5cm}} 
        }
        \rowcolor{tablegray1} 
        \multicolumn{1}{>{\centering\arraybackslash}p{0.2\linewidth}  @{\hspace{0.5cm}}}{\textbf{Teilziel}}  & \multicolumn{1}{@{\hspace{0cm}}>{\centering\arraybackslash}p{0.3\linewidth} @{\hspace{0.5cm}}}{\textbf{inhaltsbezogene Kompetenzen}} & \multicolumn{1}{@{\hspace{0cm}} >{\centering\arraybackslash}p{0.4\linewidth} @{\hspace{0.5cm}}}{\textbf{prozessbezogene Kompetenzen}} \\
        \\[-5ex]
        \rowcolor{tablegray2}
        Versuchsaufbau & Größen messen \newline Volumen berechnen & Realsituation analysieren \newline  und aufbereiten\\
        \\[-5ex]
        \rowcolor{tablegray1}
        Tabelle + Graph & Datenaufnahme Tabelle  \newline Datenaufnahme KOS \newline Darstellungswechsel & Mathematisieren der Realsituation\\
        \\[-5ex]
        \rowcolor{tablegray2}
        grafische Lösung& Funktionsgraph \newline Regression \newline Sachverhalte ablesen \newline Dreisatz \newline Darstellungswechsel  & Arbeiten im \newline mathematischen Modell \newline Modellergebnisse interpretieren \newline und validieren\\
        \\[-5ex]
        \rowcolor{tablegray1}
        Präsentation + Sicherung & Proportionalität erkennen \newline eindeutige Zuordnung  & Lösungsweg und Ergebniss darstellen \newline Fachsprache korrekt verwenden \newline Vermutungen begründet äußern\\
    \end{tabularx}
    \caption{Das Schwerpunktziel, das Erfahren der Bedeutung von Funktionen zur Lösung von Alltagsproblemen, setzt sich aus vier Teilzielen zusammen. }
    \label{table:2}
\end{table}

\end{document}
