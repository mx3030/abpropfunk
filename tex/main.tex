\documentclass[12pt,a4paper]{scrartcl}

\usepackage[T1]{fontenc}
\usepackage[utf8]{inputenc}
\usepackage[ngerman]{babel}
\usepackage{lmodern}

\usepackage{geometry}
 \geometry{
 %left=20mm,
 %right=20mm,
 top=30mm,
 }

\usepackage{amsmath}
\usepackage{amsfonts}
\usepackage[dvipsnames, table]{xcolor}
\usepackage{graphicx}
\usepackage{tcolorbox}
\usepackage{twemojis}

% Tabellen
\usepackage{tabularx}
\usepackage{multirow}
\usepackage{rotating}
\usepackage{lscape}
\usepackage{tikz}
\usepackage{makecell}

% manage file splitting
\usepackage{subfiles}

\graphicspath{{img/}{img/ink/}}

% my definitions
\definecolor{tablegray1}{RGB}{217, 217, 217}
\definecolor{tablegray2}{RGB}{242, 242, 242}
\renewcommand{\arraystretch}{2}   
\newcommand\tikzmark[1]{\tikz[remember picture,overlay] \node (#1) {};}

\title{Unterrichtsentwurf - Einführung in proportionale Funktionen durch Schülerexperimente}
\author{Maximilian Frank}

\begin{document}

\maketitle
\tableofcontents

\newpage

\subfile{sections/Unterrichtseinheit/Unterrichtseinheit}
%\newpage
\subfile{sections/Verlaufsplan/Verlaufsplan}
%\newpage
\subfile{sections/Rahmenbedingungen/Rahmenbedingungen}
%\newpage
\subfile{sections/Aufgabenanalyse/Aufgabenanalyse}
%\newpage
\subfile{sections/Didaktik/Didaktik}
%\newpage
\subfile{sections/Ziele/Ziele}
%\newpage
\subfile{sections/Methodik/Methodik}
%\newpage

\bibliographystyle{plain}
\bibliography{bibliography.bib}

\end{document}
